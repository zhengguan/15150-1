\documentclass[11pt]{article}

\usepackage{amsmath}
\usepackage{amssymb}
\usepackage{fancyhdr}

\oddsidemargin0cm
\topmargin-2cm     %I recommend adding these three lines to increase the
\textwidth16.5cm   %amount of usable space on the page (and save trees)
\textheight23.5cm

\newcommand{\question}[2] {\vspace{.25in} \hrule\vspace{0.5em}
\noindent{\bf #1: #2} \vspace{0.5em}
\hrule \vspace{.10in}}
\renewcommand{\part}[1] {\vspace{.10in} {\bf (#1)}}

\newcommand{\sml}[1]{\texttt{#1}}
\newcommand{\hasType}[2]{#1: #2}
\newcommand{\step}{\ensuremath{\Rightarrow}}
\newcommand{\stepstar}{\ensuremath{\step ^*}}
\newcommand{\stepplus}{\ensuremath{\step ^+}}

\newcommand{\myname}{Your Name}
\newcommand{\myandrew}{your-andrew-id@andrew.cmu.edu}
\newcommand{\myhwnum}{2}

\pagestyle{fancyplain}
\lhead{\fancyplain{}{\textbf{HW\myhwnum}}}      % Note the different brackets!
\rhead{\fancyplain{}{\myname\\ \myandrew}}
\chead{\fancyplain{}{15-150}}

\begin{document}

\medskip

\thispagestyle{plain}
\begin{center}                  % Center the following lines
{\Large 15-150 Assignment \myhwnum} \\
\myname \\
\myandrew \\
Day Month Year\\
\end{center}

\question{Section}{2}
\medskip
\noindent\part{6}

\begin{align*}
&\quad\;\; \sml{decimal (5 + 5)}\\
&= \ldots && \text{Justifications} \\
&= [0, 1]
\end{align*}

\medskip
\noindent\part{7}

\begin{align*}
&\quad\;\; \sml{decimal (5 + 5)}\\
&\stepstar \ldots \\
&\stepstar [0, 1]
\end{align*}

\question{Section}{4}
\medskip
\noindent\part{1}
\paragraph{Theorem:}
For all integer lists \sml{L} such that \sml{L} is \sml{all zeroes}, \sml{eval L} = \sml{0}.\\\\

\noindent The proof is by structural induction on \sml{L}.\\\\
\textbf{Base Case:} \quad  Prove for $\sml{L}=\sml{[]}$\\
\textbf{Need to show:} \\
Showing:
\begin{align*}
&\quad\;\;  \sml{eval []}\\
& \stepstar \ldots && \text{Justifications}
\end{align*}
\\\\
\textbf{Inductive Step:} \quad Prove for \sml{L} = \sml{x::R}\\
\textbf{Inductive Hypothesis:} \\
\textbf{Need to show:} \\
Showing:
\begin{align*}
&\quad\;\; \sml{eval (x::R)}\\ 
& \stepstar \ldots && \text{Justifications}
\end{align*}
\\
By the Base Case and Inductive Step, the claim is true.\\\\

\noindent\part{2}\\
\paragraph{Theorem:} 
For all natural numbers \sml{n}, \sml{sumOdd n} = \sml{n*n}.\\\\

\noindent The proof is by structural induction on \sml{n}.

\noindent\textbf{Base Case:}\quad Prove for \sml{n=0}.\\
\textbf{Need to Show:} \\
Showing:
\begin{align*}
&\quad\;\;  \sml{sumOdd 0}\\
&= \ldots && \text{Justifications}
\end{align*}
\\\\
\noindent\textbf{Induction Step:} \quad Prove for \sml{n+1}\\
\textbf{Inductive Hypothesis:} \\
\textbf{Need to show:} \\
Showing:
\begin{align*}
&\quad\;\;  \sml{sumOdd (n+1)}\\
&= \ldots && \text{Justifications}
\end{align*}
\\
Thus we have shown for all $n$, \sml{sumOdd(n)} = \sml{n*n}.\\
\end{document}
